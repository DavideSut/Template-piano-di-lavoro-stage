%----------------------------------------------------------------------------------------
%   USEFUL COMMANDS
%----------------------------------------------------------------------------------------

\newcommand{\dipartimento}{Dipartimento di Matematica ``Tullio Levi-Civita''}

%----------------------------------------------------------------------------------------
% 	USER DATA
%----------------------------------------------------------------------------------------

% Data di approvazione del piano da parte del tutor interno; nel formato GG Mese AAAA
% compilare inserendo al posto di GG 2 cifre per il giorno, e al posto di 
% AAAA 4 cifre per l'anno
\newcommand{\dataApprovazione}{GG Ottobre 2022} % da inserire

% Dati dello Studente
\newcommand{\nomeStudente}{Davide}
\newcommand{\cognomeStudente}{Sut}
\newcommand{\matricolaStudente}{1201267}
\newcommand{\emailStudente}{davide.sut@studenti.unipd.it}
\newcommand{\telStudente}{+ 39 331 35 31 426}

% Dati del Tutor Aziendale
\newcommand{\nomeTutorAziendale}{Fabio}
\newcommand{\cognomeTutorAziendale}{Pallaro}
\newcommand{\emailTutorAziendale}{f.pallaro@synclab.it}
\newcommand{\telTutorAziendale}{+ 39 333 13 68 500}
\newcommand{\ruoloTutorAziendale}{} % da inserire

% Dati dell'Azienda
\newcommand{\ragioneSocAzienda}{Sync Lab S.r.l}
\newcommand{\indirizzoAzienda}{Galleria Spagna 28, Padova (PD)}
\newcommand{\sitoAzienda}{www.synclab.it}
\newcommand{\emailAzienda}{mail@mail.it} % da inserire
\newcommand{\partitaIVAAzienda}{P.IVA 12345678999} % da inserire

% Dati del Tutor Interno (Docente)
\newcommand{\titoloTutorInterno}{Prof.}
\newcommand{\nomeTutorInterno}{NomeDocente} % da inserire
\newcommand{\cognomeTutorInterno}{CognomeDocente} % da inserire

\newcommand{\prospettoSettimanale}{
     % Personalizzare indicando in lista, i vari task settimana per settimana
     % sostituire a XX il totale ore della settimana
    \begin{itemize}
        \item \textbf{Prima Settimana - Formazione (40 ore)}
        \begin{itemize}
            \item Incontro con persone coinvolte nel progetto per discutere i requisiti e le richieste
            relativamente al sistema da sviluppare;
            \item Verifica credenziali e strumenti di lavoro assegnati;
            \item Presa visione dell’infrastruttura esistente;
            \item Formazione sulle tecnologie adottate;
        \end{itemize}
        \item \textbf{Seconda Settimana - Formazione (40 ore)} 
        \begin{itemize}
            \item Ripasso Javascript
            \item Studio framework AngularJS;
        \end{itemize}
        \item \textbf{Terza Settimana - Formazione (40 ore)} 
        \begin{itemize}
            \item Studio toolkit Ionic;
        \end{itemize}
        \item \textbf{Quarta Settimana - Sviluppo (40 ore)} 
        \begin{itemize}
            \item ;
        \end{itemize}
        \item \textbf{Quinta Settimana - Sviluppo (40 ore)} 
        \begin{itemize}
            \item ;
        \end{itemize}
        \item \textbf{Sesta Settimana - Sviluppo (40 ore)} 
        \begin{itemize}
            \item ;
        \end{itemize}
        \item \textbf{Settima Settimana - Sviluppo (40 ore)} 
        \begin{itemize}
            \item ;
        \end{itemize}
        \item \textbf{Ottava Settimana - Conclusione (30 ore)} 
        \begin{itemize}
            \item ;
        \end{itemize}
    \end{itemize}
}

% Indicare il totale complessivo (deve essere compreso tra le 300 e le 320 ore)
\newcommand{\totaleOre}{}

\newcommand{\obiettiviObbligatori}{
     \item \underline{\textit{O01}}: Acquisizione competenze sulle tematiche sopra descritte;
     \item \underline{\textit{O02}}: Capacità di raggiungere gli obiettivi richiesti in autonomia;	 
}

\newcommand{\obiettiviDesiderabili}{
	 \item \underline{\textit{D01}}: primo obiettivo;
	 \item \underline{\textit{D02}}: secondo obiettivo;
}

\newcommand{\obiettiviFacoltativi}{
	 \item \underline{\textit{F01}}: primo obiettivo;
	 \item \underline{\textit{F02}}: secondo obiettivo;
	 \item \underline{\textit{F03}}: terzo obiettivo;
}